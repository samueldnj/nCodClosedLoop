\documentclass[12pt,]{article}
\usepackage[letterpaper, margin = 1in, footskip = 0.25in]{geometry}
\usepackage{amsmath,amssymb,setspace,mathabx}
\usepackage{graphicx}
\usepackage{xcolor}
\usepackage{lscape}
\usepackage{verbatim}
\usepackage{rotating}
\usepackage[left]{lineno}
\doublespacing

\font\myfont=cmr12 at 15pt

\providecommand{\tightlist}{%
  \setlength{\itemsep}{0pt}\setlength{\parskip}{0pt}}

% Redefine \includegraphics so that, unless explicit options are
% given, the image width will not exceed the width of the page.
% Images get their normal width if they fit onto the page, but
% are scaled down if they would overflow the margins.
\makeatletter
\def\ScaleIfNeeded{%
  \ifdim\Gin@nat@width>\linewidth
    \linewidth
  \else
    \Gin@nat@width
  \fi
}
\makeatother

\let\Oldincludegraphics\includegraphics
{%
 \catcode`\@=11\relax%
 \gdef\includegraphics{\@ifnextchar[{\Oldincludegraphics}{\Oldincludegraphics[width=\ScaleIfNeeded]}}%
}%

\setcounter{secnumdepth}{5}

\ifxetex
  \usepackage[setpagesize=false, % page size defined by xetex
              unicode=false, % unicode breaks when used with xetex
              xetex]{hyperref}
\else
  \usepackage[unicode=true]{hyperref}
\fi
\hypersetup{breaklinks=true,
            bookmarks=true,
            pdfauthor={},
            pdftitle={Evaluating a precautionary harvest strategy in the presence of uncertainty in natural mortality dynamics and structure.},
            colorlinks=true,
            citecolor=blue,
            urlcolor=blue,
            linkcolor=magenta,
            pdfborder={0 0 0}}
\urlstyle{same}  % don't use monospace font for urls


\title{{\textbf{\myfont Evaluating a precautionary harvest strategy in the presence of
uncertainty in natural mortality dynamics and structure.}}}

\author{    Samuel D. N. Johnson, 
        Ashleen J. Benson, 
        Sean P. Cox, 
     \\
        Simon Fraser University, 
        Simon Fraser University, 
        Simon Fraser University, 
     \\
        samuelj@sfu.ca, 
        ajbenson@sfu.ca, 
        spcox@sfu.ca, 
     \\
}

\date{\today}


\begin{document}
\maketitle

\linenumbers

\newpage
\abstract{Canada's 2J3KL northern cod stock is rebuilding from a severely depleted
state. Recent stock assessments indicate that brief periods of extremely
high natural mortality rates for adult cod can lead to stock collapse
even in the presence of theoretically conservative fishing mortality
rates (e.g., \(F_{0.1}\)). Future non-stationarity in natural mortality
and recruitment rates of cod, as well as potential errors in assessment
of stock abundance, should therefore be taken into account when
evaluating future rebuilding strategies. This paper evaluates possible
harvest strategies for Canada's 2J3KL northern cod fishery using a
closed loop simulation framework. Our simulation framework models the
northern cod fishery system closely, by conditioning on historical
recruitment, natural mortality and growth rates estimated in the 2014
stock assessment. We include realistic assessment errors in our
projections by simulating a statistical catch at age stock assessment
model~as part of the management procedure. Candidate harvest strategies
are tested against 12 scenarios varying recruitment and mortality
dynamics, including density dependent and independent natural
mortality~scenarios. We demonstrate that a harvest strategy using an
equilibrium \(F_{0.1}\) target harvest rate may not be sufficiently
conservative in the presence of future mortality and recruitment
uncertainty and non-stationarity, and that the demographic rates largely
determine the time for rebuilding, not the choice of management
procedure.}




\newpage

{
\hypersetup{linkcolor=black}
\setcounter{tocdepth}{3}
\tableofcontents
}
\section{Introduction}\label{introduction}

Canada's northern cod stock (NAFO areas 2J3KL) appears to be rebuilding
from a severely depleted historical state (Cadigan 2015). The original
collapse of this stock in the early 1990s was initially attributed to
fishing; however, in hindsight, it appears that elevated natural
mortality could have been a key contributing factor to the rapid decline
in stock biomass over a short time. A return of natural mortality to
levels observed prior to the collapse, combined with relatively strong
recruitment, have led to sustained recent biomass growth. Continued
biomass growth at this pace could potentially re-establish directed
commercial cod fisheries in the near future.

Renewing the fishery for northern cod requires a harvest strategy
designed to take into account recognised uncertainties in cod stock
abundance and dynamics, as well as fishery economic preferences. Given
the historical context, there is considerable need for diligence in
thoroughly testing harvest strategies for weaknesses in the face of
identified cod dynamics\ldots{}.

\begin{itemize}
\tightlist
\item
  political climate: should we mention the current inshore fisheries?
  need for economic stimulation in NFL and NS?
\item
  This is where we should reference HCR papers, eg Punt (2010), Kvamsdal
  et al. (2016),
\item
  stakeholder driven options: Cox and Kronlund (2008)
\item
  Foreshadow the question here: HCRs are meant to balance the opposing
  needs of conservation and yield, and implement the PA in a rules-based
  paradigm
\end{itemize}

Closed-loop computer simulations are currently the only practical way to
test whether harvest strategy designs that appear precautionary in
theory are actually likely to be precautionary in practice (Cox et al.
2011, Wetzel and Punt 2017). Modern stock assessment models, regardless
of complexity, are not capable of providing the information needed to
design such harvest strategies (Smith 1994, Punt et al. 2016). This
limitation is because stock assessment models do not adequately account
for feedbacks between future management procedures (i.e., combinations
of data, assessment, and harvest control rule) and performance measures
related to stock biomass and fishery outcomes (e.g., yield, variability
in yield, and risk).

\begin{itemize}
\tightlist
\item
  Something about why 1 step ahead residuals are limited in use here -
  need to rank MPs, not predict the stock behaviour
\item
  Descriptive vs.~explanatory models? This probably just muddies the
  issue
\end{itemize}

Stock assessment model errors and target fishing mortality rates are the
main factors that interact to create both short-term and long-term
weaknesses in fishery harvest strategies \textbf{\emph{{[}citation
needed{]}}}. The limitations of fishery stock assessment models are
reasonably well-understood -- models that are based on high-quality data
are generally good at estimating relative changes in abundance over time
and sometimes abundance relative to reference points such as the
unfished biomass; however, the models are also not always capable of
providing unbiased estimates of the absolute fish stock biomass,
mortality rates, or productivity. For output quota fisheries, including
the northern cod fishery, biomass estimation biases from stock
assessment models are translated directly into biases in short-term
harvest quotas; that is, over-estimating biomass causes the actual
fishing mortality rate on the stock to be greater than intended.

If a fishery is managed using standard fishing mortality targets such as
\(F_{0.1}\) or \(F_{MSY}\), then biomass estimation errors can propagate
to substantially higher quotas than intended {[}\textbf{citation}{]}.
Persistent biomass over-estimation during a stock decline may lead to a
positive feedback because stock assessment model biases are usually
worst when stock biomass is changing rapidly. Thus, stock assessment
biases and relatively high target \(F\) values can lead to long-term
declines and, in some cases, collapses of important fish stocks despite
considerable efforts put into data collection, stock assessment
modelling, and theoretically conservative harvest control rules.

This paper evaluates whether a precautionary harvest strategy for 2J3KL
cod is actually precautionary under time-varying demographic rates using
a closed loop simulation framework. The key elements of the simulation
framework attempt to closely mimic northern cod stock dynamics, stock
assessment model performance, and management decision rules that aim to
promote simultaneous rebuilding of both the stock and the fishery.
Population dynamics in the historical period are matched to 2015 stock
assessment outputs (Cadigan 2015), and the harvest strategy used in the
projection is modeled on the standard precautionary harvest strategy
recommended by the Department of Fisheries and Oceans, Canada (DFO
2006).

\section{Methods}\label{methods}

Closed loop management strategy simulations for output quota fisheries
require three main components: (i) an operating model to represent
population dynamics of the stock, the mechanisms generating survey and
age-composition data, and relationships between harvest decisions and
fishing mortality on the stock; (ii) a management procedure consisting
of monitoring data, stock assessment analyses, and harvest control rules
for setting target fishing mortality and catch limits; and (iii)
performance indicators for comparing simulated outcomes against fishery
objectives. The following sections describe the models used for each of
these components for our Northern Cod simulations.

\textbf{Is the following part really necessary? Shouldn't we just have a
table of parameter definitions?}

Our model notation attempts to maintain consistent conventions for state
variables and parameters across both the operating model and stock
assessment model, while also making clear the differences between
operating model variables, equilibrium solutions, parameters estimated
in stock assessment models, and variables derived from these parameter
estimates. As a general rule, any parameter or variable (e.g., \(B_0\))
that does not show a \^{} or \(\sim\) symbol is part of the operating
model. Variables without subscripts for time (e.g.,
\(F_{MSY}, B_{MSY}, B_0\)) are considered constant and usually represent
equilibrium quantities. The symbol \^{} over a variable indicates a
parameter (e.g., \(\hat{B}_0\) ) or variable estimated by the stock
assessment model. The combination of \^{} and \(\sim\) symbols and time
subscripts (e.g., \(\hat{\tilde{B}}_{t,MSY}\) ) indicates a quantity
that is a function of estimated stock assessment model parameters while
time subscripts (e.g., \(T\)) on parameters such as the one shown above
indicate an estimate of that quantity given data up to the time step
indicated. Vector objects are denoted using R-like notation such as
\(1:T\) in subscripts (e.g., \(\hat{B}_{1:T}\) ).

\subsection{Age-structured operating
model}\label{age-structured-operating-model}

\subsubsection{Population dynamics}\label{population-dynamics}

Abundance dynamics were simulated via an age-structured model with
\(A = 14\) age classes, where the index \(A\) represents a plus-group.
Notation, parameter settings, and equations for the operating model are
given in Tables 1, 2, and 3, respectively. The total simulation time
frame (1983-2035) is divided into an historical period,
\(t \leq T_1 - 1\), corresponding to 1983-2014, and a projection period,
\(T_1 \leq t \leq T_2\), corresponding to 2015-2035. The operating model
population is initialized in the deterministic, unfished equilibrium
state at time t~=~1 (corresponding to the year 1983). For northern cod,
these initial equilibrium abundances are then modified by age-specific
multipliers to re-scale abundances to non-equilibrium numbers-at-age
taken from the 2015 NCAM stock assessment (Cadigan 2015). State dynamics
are then driven by stochastic growth, recruitment, and mortality
processes.

\subsubsection{Growth, recruitment, natural and fishing mortality, and
maturity}\label{growth-recruitment-natural-and-fishing-mortality-and-maturity}

We modelled size-at-age of cod by cohort using cohort-specific
Ford-Walford growth parameters \((\alpha_t,\rho_t)\) estimated from the
historical size-at-age data {[}citation needed{]}. These were estimated
via ordinary linear regression of length-at-age \(a\) on length-at-age
\(a-1\). The \((\alpha_t,\rho_t)\) values were highly correlated, so we
modelled \(\rho_t\) as a linear function of \(\alpha_t\) (OM2.10) to
ensure that simulated future growth parameters maintained a similar
correlation to the empirical data \textbf{\emph{We should probably show
this relationship}}. Equations OM2.11-2.13 show the sequence of
calculations used to derive the annual weights-at-age. Length-weight
conversion parameters \(c_1 = 7.59 \cdot 10^{-5}\) and \(c_2 = 3.06\)
were obtained from (Froese et al. 2014).

Recruitment to the population is assumed to occur in a single pulse at
the beginning of the year. Annual age-1 recruitment values in the
historical period are derived from the NCAM annual values of age-2
recruitment and the age-2 natural mortality rate estimated for 1983,
assuming age-1 mortality for the historical period is identical to age-2
mortality, e.g. \(R_t = N_{2,t+1} e^{M_2,t}\). Similarly, natural
mortality rates for the historical period were fixed at
age-/year-specific estimates from NCAM.

Projections of the Walford growth parameter \(\alpha_t\), recruitment
and natural mortality are modelled as AR(1) processes in the operating
model. Equation OM2.14, where the generic variable X represents one of
these processes, gives the general formula. Autocorrelated deviations in
the Walford growth intercept \(\alpha_t\) have a standard deviation of
\(\sigma_\alpha=0.1\) and an autocorrelation coefficient of
\(\gamma_\alpha=0.2\). Recruitment deviations have a standard deviation
\(\sigma_R = 0.428\), to maintain consistency with variability in NCAM,
and an autocorrelation coefficient of \(\gamma_R=0.8\). Finally, natural
mortality deviations have a standard deviation of \(\sigma_M = 0.255\)
and autocorrelation coefficient of \(\gamma_M=0.534\), consistent with
NCAM estimates. Simulated natural mortality and growth rates for the
projection period were scaled such that values of the historical and
projection periods match exactly at the end of the historical period
(\(t = T_1-1\)).

{[}Revise the following paragraph{]}

Equations OM2.17-2.20 gives the abundance-at-age, spawning biomass, and
exploitable biomasses implied by the parameters and fishing mortality
rates. The operating model assumes a single fishery with \(F_{a,t}\)
values derived from NCAM in the historical period. In the projection
period, fully selected \(F_t\) values are found as solutions to the
catch equation (OM2.21) given annual output quotas from the management
procedures, with selectivity functions derived from NCAM \(F_{a,t}\) in
the historical period. Fishing selectivity-at-age is time varying, which
in the historical period is solved from NCAM fishing mortality-at-age
estimates by scaling the maximum values to 1. Projected
selectivity-at-age is resampled with replacement from the historical
values using a time series bootstrap, which is similar to a traditional
bootstrap with one important difference. Where a traditional bootstrap
will sample single points of data with replaced, the time series
bootstrap samples random length segments from the history in order to
preserve any auto-correlation that may exist in those segments.

Maturity-at-age is modeled parametrically, by cohort, using logistic
functions (equation OM 2.2). In the historical period, cohort-specific
age-at-maturity ogives for 50\% and 95\% mature are estimated from
observations of the proportion mature at age in the RV survey. When
projecting forward, ogives are resampled from the historical data using
a time series bootstrap {[}\textbf{definition or citation}{]}.

\subsubsection{Data generation from the operating
model}\label{data-generation-from-the-operating-model}

At each time step, the operating model generates a log-normally
distributed survey biomass index with catchability coefficient (OM2.23)
and vectors of observed age-proportions using both fishery and survey
selectivity curves. The survey index standard error is assumed constant
at \(\tau_I = 0.3\). Age-composition is modelled using multivariate
logistic distributions with independent errors (OM2.24-2.26) (Schnute
and Richards 1995). Simulated ageing errors had a standard deviation of
\(\tau_P=0.2\).

\subsubsection{Operating model projection
scenarios}\label{operating-model-projection-scenarios}

Cod production is highly sensitive to adult natural mortality rates (M)
and age-1 recruitment. We defined 12 operating model scenarios based on
combinations of four natural mortality and three recruitment assumptions
in the projection period.

The first natural mortality scenario (conM) is a stationary, zero-trend
random walk (\(\sigma_M = 0.255\)) around the historical mean M value.
This scenario provides a best-case benchmark. Other natural mortality
scenarios assumed that historical natural mortality patterns resulted
from the above random walk plus short-term pulses of extreme mortality,
occurring at random with frequencies once every 40 years (pM40), once
every 20 years (pM20), and a density-dependent pulse occuring every 20
years while spawning stock biomass is below \(B_{lim}\) (pM20lim). A
pulse magnitude of 650\% of average was chosen to mimic the highest
observed M event in the historical period. {[}\textbf{Should we
reference a citation for density dependent mortality?}{]}

{[}\emph{Dial up recruitment to BH?? At least in projection
period\ldots{}}{]}

The three recruitment scenarios assume that future average is (i)
constant at the recent average recruitment from 2005 -- 2014, which is
16\% of the 1980s mean recruitment (\(.16\bar{R}\)), (ii) 50\% of 1980s
mean recruitment (\(.5\bar{R}\)), or (iii) an increasing trend from the
recent average to half the 1980s average (\(inc\bar{R}\)).

While this limited suite of scenarios is far from exhaustive, it
suffices to demonstrate some of the challenges in developing management
procedures in the presence of non-stationary population dynamics and in
judging performance with respect to limit reference points and fishery
objectives.

\subsection{Management procedures}\label{management-procedures}

Simulated management procedures (MPs) in the projection period consist
of three components: (1) a fishery data set involving time-series
(\(t = 1,2,\dots,T\)) of total catch, an exploitable biomass index
time-series, and proportions-at-age in the fishery catch and research
vessel (RV) survey; (2) a stock assessment model that uses the simulated
data to estimate historical biomass, recruitment, natural mortality,
selectivity, and stock-recruitment parameters up to time step \(t\)
(AM.1), as well as any values required by harvest control rules (Cox et
al. 2013); and (3) a harvest control rule for computing a catch limit
based on stock assessment results. The sections below describe how each
of these components is implemented in the simulations.

\subsubsection{Simulated stock assessment
data}\label{simulated-stock-assessment-data}

Although the operating model simulates the data used in fishery stock
assessments, the MP controls the types, frequency, and precision of the
simulated data because these are typically under management control.
Annual estimates of cod spawning biomass are required by all management
procedures. For this study, we generated unbiased, absolute values of
spawning biomass as the biomass index data (OM2.23). The coefficients of
variation (CVs) of these estimates were constant over time and set to
values estimated in the 2015 stock assessment (see above for standard
errors). Fishery and survey age-composition data required for the
simulated SCA stock assessments (defined below) are generated annually
from OM2.24-2.26.

\subsubsection{Catch-at-age stock assessment
models}\label{catch-at-age-stock-assessment-models}

The statistical catch-at-age assessment model (AM; Table 4) used in the
simulated management procedures is defined in AD Model builder and
differs slightly from the age-structured operating model (Fournier et
al. 2012). The four main differences are that (i) recruitment in the AM
is based on a Beverton-Holt stock-recruit relationship with uncorrelated
process errors (AM.6 and Table 6 eq L.4), (ii) catch in the AM is taken
assuming a discrete fishery (i.e.~a single fleet) occurring at the
beginning of the year (AM.7) instead of continuously as it is in the
operating model, (iii) weight-at-age is assumed constant in the AM, and
(iv) the AM assumes only a single time-varying \(M_t\) value that
applies to all ages. Equations AM.1-AM.8 show how the relevant
calculations in the AM are affected by these differences. The AM
estimator makes use of data from the OM including catch, biomass survey
indices, and proportions-at-age in the catch and survey. Operating model
schedules of maturity-at-age are assumed constant and known in the AM
and are therefore part of the assessment input data. Recruitment
deviations are only estimated for years \(t = 2,3,...,T - a_{50}^{mat}\)
because there is little information in age-composition data about more
recent recruitment. We use age-at-50\% maturity instead of age-at-50\%
selectivity to bound the size of the recruitment deviation vector
because the former is a known input whereas the latter is based on
estimated model parameters and therefore violates AD Model Builder rules
of automatic differentiation (i.e., the length of a parameter vector
cannot be a function of an estimated parameter). Natural mortality rate
is estimated in the AM as a random walk to allow for non-stationary
natural mortality. In all cases, we use a somewhat informative prior on
the initial \(M_t\) value at t~=~1.

Maximum likelihood estimates of error variances are computed
analytically in the AM by conditioning on the leading parameters. For
this study, we assumed that RV survey catchability \(q = 1\) in the AM
for two reasons. First, preliminary tests of the AM estimator assuming
\(q = 1\) closely matched actual stock assessments that estimated \(q\)
(which was close to 1 anyway); and second, assuming \(q = 1\) gives a
more stable AM estimator in closed-loop simulations.

Table 6 provides the likelihood components and calculations involved in
the negative-log-posterior distribution function (\(G\); L.10). The AM
uses an errors-in-variables (EIV) maximum likelihood formulation for
modeling the combined biomass index and process error likelihood
(\(l_{IR}\); L.1-L.6). The EIV approach reduces the number of estimated
parameters by assuming a total error variance (\(\kappa^2\)) that
comprises observation error (\(\tau_I^2\)) and age-1 recruitment process
error (\(\sigma_R^2\)) components, i.e.,
\(\kappa^2 = \sigma_R^2 + \tau_I^2\). Assuming that the observation
error proportion of this total is known (\(\rho_{CAA} = 0.1\)), the
individual variance estimates are \(\tau_I^2 = \rho_{CAA}\kappa^2\) and
\(\sigma_R^2 = (1 - \rho_{CAA})\kappa^2\), where the estimate of the
total variance \(\kappa^2\) is given by L.5 (Schnute and Richards 1995).
Our justification for the EIV likelihood is similar to our \(q=1\)
assumption; that is, it is generally faster to simulate and produces
results similar to more complex and time-consuming estimation methods.

We use a robust normal likelihood (Fournier et al. 1998) for the
age-proportion data (L.7) assuming sample sizes are all equal to an
effective size \(n = 50\). The total negative log-posterior distribution
function includes an informative Beta prior distribution on the
stock-recruitment steepness parameter (\(h\); L.8) and an informative
\(N(0.2,0.05^2)\) prior distribution on the natural mortality rate at
\(t=1\). The shape parameters \((\beta_1,\beta_2)\) of the Beta
distribution (L.8) for steepness are derived via moment matching to a
prior mean (\(\mu_h = 0.7\)) and standard deviation
(\(\sigma_h = 0.08\)) given the constraint \(0.2 < h < 1\). These
informative prior distributions improve stability of the AM parameter
estimation procedure, but otherwise have little impacts because
simulated harvest control rules do not use equilibrium yield-per-recruit
estimates.

\subsubsection{Harvest control rule}\label{harvest-control-rule}

We examined a four management procedures that differed by the harvest
control decision rule, with two variable \(F\) rules and two constant
\(F\) rules. The first variable \(F\) procedure is a ``hockey-stick''
rule with a constant target fishing mortality rate of
\(F_{0.1} = 0.18/yr\) (Cadigan 2015). This rule, labeled F0.1, uses the
estimated present state of the stock (i.e., \(t = T\)) from the AM and a
projected expected biomass to determine a catch limit \(Q_{T+1}\) for
the upcoming year using a 2-stage precautionary harvest control rule
(Figure 1):

\begin{equation}
\hat{F}_{T+1} = \left\{ \begin{array}{ll}
                  F_{lim} \frac{\hat{B}_{T+1}}{B_{lim}} & \hat{B}_{T+1} < B_{lim}, \\
                  F_{lim} + (F_{0.1} - F_{lim}) \frac{\hat{B}_{T+1} - B_{lim}}{B_{lim}} & B_{lim} \leq \hat{B}_{T+1} < 2 B_{lim}, \\
                  F_{0.1} & \hat{B}_{T+1} \geq 2 B_{lim}. 
                  \end{array} \right.
\end{equation}

\noindent This harvest rule is defined to match the decision process
used in the northern cod fishery, and modifies the harvest strategy that
Fisheries and Oceans, Canada, suggests to comply with the precautionary
approach to fisheries management (DFO 2006). The modification is
intended to re-establish a fishery as the stock recovers from
historically low levels, and adds the secondary ramp from 0 to the lower
reference harvest rate (\(F_{lim} = 0.05\)) for the critical zone
\(0 \leq \hat{B_T} \leq B_{lim}\). To define the critical zone, we use
the limit reference biomass \(B_{lim} = 885kt\), which is the average
biomass during the 1980s as estimated by NCAM. The upper stock
reference, which defines the upper limit of the cautionary zone, is
defined as twice the limit reference point,
\(2\cdot B_{lim} =  1770kt\). At the upper stock reference, the control
rule sets the target fishing mortality rate as \(F_{0.1} = 0.18\), which
is scaled linearly from \(F_{lim}\) within the cautionary zone.

The second variable \(F\) rule modifies the F0.1 MP by replacing the
secondary ramp in the critical zone with a stepped, fixed TAC rule.
Under the stepped TAC rule (steppedTAC), the annual quota is limited by
the following rules: {[}\textbf{\emph{Since this is just research, I
changed the step function by shifting everything to the right by one
step, i.e.~0TAC in 0-25\%\(B_{lim}\), needs new plot}}{]}

\begin{enumerate}
\def\labelenumi{\arabic{enumi}.}
\tightlist
\item
  \(Q_{T+1} = 0\) when \(B_{T+1}\) is 0\%-25\% of \(B_{lim}\),
\item
  \(Q_{T+1} = 5,000t\) when \(B_{T+1}\) is 25.1\%-50\% of \(B_{lim}\),
\item
  \(Q_{T+1} = 10,000\) when \(B_{T+1}\) when SSB is 50.1\%-75\% of
  \(B_{lim}\),
\item
  \(Q_{T+1} = 15,000\) when \(B_{T+1}\) when SSB is 75.1\%-100\% of
  \(B_{lim}\).
\end{enumerate}

This modification is intended increase the conservation performance of
the F0.1 rule by reducing fishing pressure inside the critical zone,
while at the same time reducing the annual variation in yield while the
stock is still very low. Furthermore, the point at which fishing is
reduced to zero is above \(B_{T+1} = 0\), which is more precautionary
than F0.1. This reduction in AAV would likely increase the stability of
cod markets for industry stakeholders, allowing for a more managed,
increase in precautionary nature will likely improve conservation
performance. {[}\textbf{\emph{We should probably indicate that GEAC
asked for something like this?}}{]}

Finally, we describe the two constant F management procedures. The
first, conF, uses the lower reference fishing mortality rate
\(F_{lim}= 0.05\). The second, F\_SAR, uses a fishing mortality rate
\(F= 0.122\) estimated from the recent 2014 TAC of 35,000t set by the
recent SAR document {[}\textbf{\emph{get citation, or just switch this
to constant F0.1}}{]}, with reference exploitable biomass assumed to be
538,000 (Personal Communication, Kris Vascotto).

Once the quota is determined by the harvest control rule, the operating
model determines the effective fishing mortality rate \(F_{T+1}\). This
is found by numerically solving the Baranov catch equation,

\begin{equation}
Q_{T+1} = \frac{\hat{F}_{T+1}}{\hat{M}_{T} + \hat{F}_{T+1}} \hat{B}_{T+1} \left(1 - e^{-(\hat{F}_{T+1} + \hat{M}_{T}) } \right),
\end{equation}

\noindent where \(\hat{B}_{T+1}\) is a 1-year-ahead stock assessment
model projection of the exploitable biomass for the coming year and
\(\hat{M}_T\) is the current estimate of natural mortality from the
assessment model. This projection is based on deterministic age-1
recruitments from the estimated spawner-recruit relationship for years
\(T - a_{50}^{mat}\) to \(T+1\) because recruitment is not
well-estimated in more recent years (see above).

Annual catch taken from the operating model population is set equal to
the TAC obtained from the target fishing mortality; that is, we assume
that all the TAC is landed (no unreported or at-sea discarding) and the
fisheries close when the TAC is reached. Some of these assumptions are
not realistic for 2J3KL cod, for example, there is probably considerable
discarding due to quota restrictions as the stock recovers
(\textbf{\emph{size problems?? - citation}}).

\subsubsection{Performance measures}\label{performance-measures}

We use five common metrics to summarise conservation and yield
performance of simulated management procedures. Conservation performance
was measured using the median proportion of simulations in which the
spawning biomass drops below the operating model \(B_lim\), i.e.,

\begin{enumerate}
\def\labelenumi{\roman{enumi}.}
\tightlist
\item
  the probability (\(p_{t,crit}\)) of spawning stock biomass being
  within Critical zone (\(B_t < B_{lim}\)) at the end of year \(t\)
  (\(t = 2017, 2019, 2024\)).
\end{enumerate}

Yield performance of each MP is summarized via:

\begin{enumerate}
\def\labelenumi{\roman{enumi}.}
\setcounter{enumi}{1}
\tightlist
\item
  the median average annual catch (\(\bar{C_t}\)) during the period
  \([2015,t]\);
\item
  average annual variability of yield (AAV):

  \begin{equation}
  AAV = \frac{\sum_{t=1}^T |Q_t - Q_{t-1}| }{\sum_{t = 1}^T Q_t},
  \end{equation}

  \noindent where Qt is the simulated quota obtained from applying a
  given MP in year \(t\).
\end{enumerate}

The final 2 metrics provide information about stock rebuilding during
the projection period:

\begin{enumerate}
\def\labelenumi{\roman{enumi}.}
\setcounter{enumi}{3}
\tightlist
\item
  first year in which \(B_{lim}\) is reached with probability p:
  \(T_{lim}^{p}\);
\item
  first year in which \(USR = 2B_{lim}\) is reached with with
  probability p: \(T_{USR}^{p}\).
\end{enumerate}

These are estimated as the first time \(T_{ref}^{p}\) that the
\((100 - p)\)th percentile of all projected \(B_t\) trajectories passes
the reference levels \(B_{lim}\) or \(USR = 2B_{lim}\), for
\(p = 50, 75, 95\).

\section{Results}\label{results}

\subsection{Simulation model dynamics}\label{simulation-model-dynamics}

The operating model (OM) agreed reasonably well with the historical
spawning biomass estimates from NCAM (Figure 3), which is not surprising
given that the OM is initialized with NCAM abundances,
natural-mortality-at-age, fishing mortality-at-age, as well as changes
in weight and maturity at age over time. Figure 3 also shows the
alternative assessment estimates of biomass, F, and M from applying the
management procedure assessment model (i.e., AM, Tables 5 and 6) to the
actual 2J3KL northern cod data. Biomass estimates from this model also
agree reasonably well with NCAM in the 1980s and in the period following
the collapse (Table 7). Although the AM also estimates high M leading up
to the collapse, it also estimates much higher F and lower M during the
collapse compared to NCAM. This probably occurs because of the AM
assumption that survey catchability \(q = 1\), while NCAM allows
temporal variation in survey \(q\).

Simulation envelopes for projected \(M\) values differ between the \(M\)
scenarios, as expected, but also between recruitment scenarios when
\(M\) was density dependent (Figure 4). The conM scenario M values occur
in a tight envelope centered on the historical mean (Figure 4 -- conM),
while the two pulse M scenarios show either one period of pulse M events
for the 40-year frequency (pM40) or 4 periods (one sustained for 3
years) for the 20-year frequency (pM20). The biomass-dependent natural
mortality pulses (pM20lim) behave differently under the different
recruitment scenarios. In the low recruitment scenarios
(pM20lim\_\(.16\bar{R}\)) the biomass stays below \(B_{lim}\) more
often, leading to aggregate behaviour similar to the 20-year frequency
scenario (pM20). In the high recruitment scenario
(pM20lim\_.5\(\bar{R}\)) pulses occur in the short term because biomass
is currently below \(B_{lim}\), and but less often near the end of the
time series, resulting in a final pulse that is sustained for only one
year.

Recruitment scenarios behave fairly consistently, in the absence of a
stock-recruitment relationship (Figure 5). The .16\(\bar{R}\) scenario R
values occur in a tight envelope around the recent (2005 -- 2014)
average age-1 recruitment for the projection period. The \(inc\bar{R}\)
scenario envelope widens as average age-1 recruitment trends towards
50\% of the 1980s mean. The \(.5\bar{R}\) scenario values occur in a
wide, uniform envelope the same width as the final year of the
\(inc\bar{R}\) scenario. Recruitment values are generated by modifying
the average R value with log normal deviations, which act proportionally
and result in wider envelopes for larger average recruitments.

We selected four example replicates to illustrate the simulated dynamics
of the conM 3 different pM operating model scenarios for average R
recruitment in Figures 6 through 9. Only replicates for the .5R
recruitment scenario are shown, as the behaviour of 1.5R scenarios is
the same but with most quantities inflated by a factor of 3. Results are
summarized using retrospective patterns of AM performance, the realized
spawning biomass, catch, and fishing mortality outcomes from the
closed-loop simulations. Results are presented only for the F0.1
management procedure, as the behaviour of the other MPs all had similar
dynamics in each replicate.

The conM operating model scenario is the most optimistic natural
mortality scenario. Simulated spawning stock biomass (SSB) increases
slowly on average during the projection period (Figure 6a), with
occasional spikes followed by declines occurring 3 years after low
recruitment events due to the maturity lag (Figure 6d). The AM shows an
interesting retrospective pattern, often underestimating SSB while the
stock is increasing, but overestimating SSB during years of steep
decline corresponding to low R events (Figure 6a, blue lines).
Overestimates of biomass during years of steep decline causes spikes in
realized fishing mortality resembling fishing mortality in 2003, much
lower than peak historic levels.

The `pulse' simulation scenarios represent the case of periodic (every
40 or 20 years) extreme 1-year natural mortality events, after which M
returns to the historical average. In these scenarios, the frequency of
high M events determines their relative impact -- more frequent extreme
events (e.g.~pM20, pM20lim) maintain lower average spawning biomass, and
thus realize a smaller stock decline than less extreme events
(e.g.~pM40) in which spawning biomass is allowed to build to high levels
(compare Figures 7a and 8a). Because of this, the pulse20 scenarios are
similar to the NCAM estimated M history, with periodic increases and
decreases in SSB (Figures 7a, 9a). In addition, recruitment affects the
SSB trajectory, with low recruitment events contributing to stock
declines, as in the conM scenario. Occasionally, recruitment will drop 3
years before the mortality pulse, combining to reduce the SSB even
further than the high M events. Due to these combined forces of
recruitment and mortality, the retrospective pattern in assessed stock
biomass creates a lag leading to large spikes in realized fishing
mortality exceeding historic levels, resulting in overfishing following
precipitous declines in SSB (Figures 7c, 8c, 9c). However, with less
frequent pulses recovery from these events is rapid due to the lower
average value of M over the projected period, as well as the lack of
autocorrelation in the mortality time series (compare Figures 7a, 8a).

Generally, recruitment and mortality can independently, and in
combination, create precipitous declines in spawning stock biomass,
leading to large assessment errors and spikes in realised fishing
mortality. This phenomenon is observable in the single replicate
retrospective analyses where small pulses of fishing mortality occur 3
years after a low recruitment event in the constant M scenario (Figure
6a), and larger pulses in the pulse M scenarios when low recruitment and
high mortality events combine (Figures 7a, 8a and 9a). The same
phenomenon is observable in the aggregate fishing mortality envelope
plots, where the pulse M scenarios have large spikes in the aggregate
performance, with higher spikes for the 1.5R scenarios (Figure 10).

In spite of the complex nature of the dynamical relationship between M
and R, the envelope plots of biomass depletion as a fraction of
``\(B_{lim}\)'' and catch show similar behaviour for each mortality
scenario (Compare Figures 11, 12, 13, 14 for .5R {[}\textbf{\emph{Surely
we can reduce the number of figures here\ldots{}}}{]}). For lower
average M scenarios (conM, pM40) the distributions in the envelopes are
somewhat more concentrated, while the higher average M scenarios (pM20,
pM20lim) have more variability. Comparing the same mortality scenarios
between 1.5R and .5R recruitment scenarios shows the same general
behaviour, but inflated by a factor of 3 as in the single replicates.

\subsection{Evaluation of management
procedures}\label{evaluation-of-management-procedures}

The ``No-fishing'' scenarios illustrate the average dynamics of the
simulated population across the range of natural mortality and
recruitment scenarios we tested (Tables 8 and 9). This provides a
benchmark for the remaining management procedures within each scenario.

\begin{itemize}
\tightlist
\item
  Rank MPs according to some metrics that we used
\item
  Show depletion scaled by NoFish - dynamic B0
\end{itemize}

The conM mortality scenario is the most conservative ``best case''
future scenario, resulting in the highest average catch for both
recruitment scenarios and best times to \(B_{lim}\) and \(2 B_{lim}\) .
The pM scenarios are the least conservative, resulting in a lower
average catch and higher probabilities of being below \(B_{lim}\) , even
in the absence of fishing (Tables 8 and 9). The introduction of
harvesting moves the system to a less conservative state in all
scenarios.

The two variable F MPs performed largely the same across scenarios.
Differences were primarily observed in the 2 yield performance criteria,
average catch and average annual variation (AAV) (Tables 8 and 9). In
the short term, maxTAC realises higher average catch than noMaxTAC due
to the TAC ceiling exceeding the HCR at low abundance (Figure 2), while
they are both similar in the long term. AAV differed the most when
mortality rates were higher on average, with maxTAC realising lower AAV
than noMaxTAC in the pulse M scenarios. The maxTAC MP was designed to
reduce AAV while SSB is at low levels, and this is evident in the
simulation results. Note however, that when explored for the full set of
simulations the aggregate difference in catch and depletion between the
MPs is minor (Figures 11, 12, 13, 14).

{[}\textbf{\emph{We should look at Cleveland plots over scenarios, scale
out by NoFish for dep/catch plots to have a scenario independent
performance}}{]}

The two constant F MPs performed similarly in dynamics. They both
tracked spawning stock biomass with stability, and the AAV was the
lowest for both MPs in all scenarios. The main difference between the
two was in average catch, with F\_SAR consistently realising about 200\%
of the conF average catch over 3, 5 and 10 years across scenarios. Due
to this, the probability of leaving the critical zone, and the time it
took to do so, were higher for F\_SAR than conF. The limit reference
point ``\(B_{lim}\)'' = (``SSB'' ) ̅\_(1983:1989) is used to define the
upper limit of the critical stock status zone at 885Kt according to NCAM
estimates (Table 7). \textbf{The main differences in performance of MPs
reaching the limit reference point ( \(T_{lim}^{50}\)⁡ ,
\(T_{lim}^{75}\), \(T_{lim}^{95}\)) or upper stock reference point
(\(T_{USR}^{50}\), \(T_{USR}^{75}\), \(T_{USR}^{95}\)) were between
natural mortality scenarios rather than between MPs (Tables 8 and 9)}.
Scenarios with lower average natural mortality were found to reach
\(B_{lim}\) more quickly, sometimes within 5 years of beginning the
management procedure. MPs in the 1.5R scenarios were the only cases
where the USR was reached with any probability during the projection
period, caused by the higher average recruitment.

\section{Discussion}\label{discussion}

In this paper, we presented a closed-loop simulation framework for
evaluating candidate harvest strategies for future management of 2J3KL
(northern) cod. This simulation framework uses operating and assessment
models that differ in complexity from the recent assessment of northern
cod because of the intensive and repetitive nature of closed-loop
simulation (Cadigan 2015). Nevertheless, the simulations effectively
demonstrate some of the key challenges in designing rebuilding
strategies for stocks that experience time-varying demographic
parameters related to natural mortality and recruitment. In cases where
time-varying mortality and recruitment matched recent historical
dynamics, even conservative fishery policies derived under equilibrium
assumptions fail to actually achieve conservative outcomes.

We ran 60 simulations, combining 5 management procedures and 12
scenarios, which were combinations of 4 mortality scenarios and 3
recruitment scenarios. We found natural mortality and recruitment of
23JKL cod were the major contributors to the simulation model dynamics
and the management outcomes as measured by the performance metrics we
defiend. These two main drivers also combined in interesting ways,
giving rise to large assessment errors and spikes in realised fishing
mortality in projected scenarios.

Recent stock assessments (Cadigan 2015) estimate large variability in
recruitment and natural mortality processes in 2J3KL cod, although the
models are largely descriptive and do not offer hypotheses to explain
such behaviour. As a result, our operating model projections of
recruitment and natural mortality dynamics mainly involve
density-independent random processes. The exception, p20Lim, represented
the hypothesis that small spawning stocks are more vulnerable to
stochastic events affecting mortality.

The importance of natural mortality and recruitment is highlighted by
the \(T_{lim}\)⁡ and \(T_{USR}\) metrics, which measure the time that
\(B_{lim}\) and \(2B_{lim}\), respectively, are first exceeded in 50\%,
75\% and 95\% of simulation replicates. Small increases in the
probability of high mortality significantly slow the pace that 23JKL cod
reaches \(B_{lim}\) , with less than half of the management procedures
in .5R scenarios reaching \(B_{lim}\) with 75\% probability. However,
the same scenarios with a 3-fold increase in average recruitment show a
marked improvement in their performance, reaching the USR with 50\%
probability within 10 years in all cases.

On its own, the relative frequency of high impact M events has a
counter-intuitive effect on spawning stock biomass. Higher frequency M
events in the pM20 and pM20lim scenarios reduce the capacity of the
stock to increase to high levels, reducing the relative magnitude of the
assessment error during declines, producing smaller spikes in fishing
mortality and less catastrophic collapses. In contrast, the lower
frequency in pM40 events allow biomass to reach high projected SSB, as
well as the largest magnitude collapses.

\subsection{Limitations}\label{limitations}

The suite of operating models examined here is not exhaustive with
respect to potential future natural mortality, recruitment, and fishing
scenarios. Clearly, we lack mechanistic understanding of the
relationships among natural mortality, fishing mortality, and stock
biomass. With the exception of the biomass-dependent 20-year pulse
\(M\), we assume that recruitment and natural mortality are
density-independent processes. While it is possible that recruitment and
M may be density-independent over the short-term, such as our 20-year
projection period, this is probably not true over multi-decadal time
scales.

Although we included a diverse set of operating model scenarios and
incorporated realistic assessment model errors in the simulated
management procedures, there are some places where the model
implementation could be adjusted. We did not account for random
implementation uncertainty, or stochastic deviations between the
intended TAC and the actual catch, and instead we assumed that the TAC
was the entire amount of removals from the stock. However, there is
potential in the fishery for unreported catch. For instance, subsistence
fishing and at-sea-discarding of smaller pieces (high-grading) may
combine in some years to increase fishing mortality to high levels, and
the associated risks should be considered.

To simplify dynamics in the projections we used a parametric selectivity
model. This model was fit to the estimated F series from the NCAM
assessment of the historical period, and therefore represents an average
selectivity model over that period. During this period, there were
changes in fishing gear when fishing vessels of different sizes targeted
the stock, accompanied by a switch in fishing practices, as well as a
large scale collapse of the resource. These changes in fishery dynamics
imply that the average selectivity model we used in the projections is
likely biased from the current fishing practices.

\subsection{Conclusion}\label{conclusion}

NOT PRECAUTIONARY, as the main driver of stock recovery to empirical
Blim is the M and R dynamics, not the MP. \textbf{\emph{Fill this in
more when the reruns are complete}}

\section*{References}\label{references}
\addcontentsline{toc}{section}{References}

\hypertarget{refs}{}
\hypertarget{ref-cadigan2015state}{}
Cadigan, N.G. 2015. A state-space stock assessment model for northern
cod, including under-reported catches and variable natural mortality
rates. Can. J. Fish. Aquat. Sci. \textbf{73}(2): 296--308. NRC Research
Press.

\hypertarget{ref-cox2011management}{}
Cox, S., Kronlund, A., and Lacko, L. 2011. Management procedures for the
multi-gear sablefish (anoplopoma fimbria) fishery in british columbia,
canada. Can. Sci. Advis. Secret. Res. Doc \textbf{62}.

\hypertarget{ref-cox2008practical}{}
Cox, S.P., and Kronlund, A.R. 2008. Practical stakeholder-driven harvest
policies for groundfish fisheries in british columbia, canada. Fish.
Res. \textbf{94}(3): 224--237. Elsevier.

\hypertarget{ref-cox2013roles}{}
Cox, S.P., Kronlund, A.R., and Benson, A.J. 2013. The roles of
biological reference points and operational control points in management
procedures for the sablefish (anoplopoma fimbria) fishery in british
columbia, canada. Environ. Conserv. \textbf{40}(4): 318--328. Cambridge
University Press.

\hypertarget{ref-DFO2006A-Harvest-Strat}{}
DFO. 2006. A harvest strategy compliant with the precautionary approach.
DFO Can. Sci. Advis. Rep.

\hypertarget{ref-fournier1998multifan}{}
Fournier, D.A., Hampton, J., and Sibert, J.R. 1998. MULTIFAN-cl: A
length-based, age-structured model for fisheries stock assessment, with
application to south pacific albacore, thunnus alalunga. Can. J. Fish.
Aquat. Sci. \textbf{55}(9): 2105--2116. NRC Research Press.

\hypertarget{ref-fournier2012ad}{}
Fournier, D.A., Skaug, H.J., Ancheta, J., Ianelli, J., Magnusson, A.,
Maunder, M.N., Nielsen, A., and Sibert, J. 2012. AD model builder: Using
automatic differentiation for statistical inference of highly
parameterized complex nonlinear models. Optimization Methods and
Software \textbf{27}(2): 233--249. Taylor \&amp; Francis.

\hypertarget{ref-froese2014bayesian}{}
Froese, R., Thorson, J.T., and Reyes, R. 2014. A bayesian approach for
estimating length-weight relationships in fishes. J. Appl. Ichthyol.
\textbf{30}(1): 78--85. Wiley Online Library.

\hypertarget{ref-kvamsdal2016harvest}{}
Kvamsdal, S.F., Eide, A., Ekerhovd, N.-A., Enberg, K., Gudmundsdottir,
A., Hoel, A.H., Mills, K.E., Mueter, F.J., Ravn-Jonsen, L., Sandal,
L.K., and others. 2016. Harvest control rules in modern fisheries
management. Elem Sci Anth \textbf{4}. University of California Press.

\hypertarget{ref-punt2010harvest}{}
Punt, A.E. 2010. Harvest control rules and fisheries management.
\emph{In} Handbook of marine fisheries conservation and management.
Oxford University Press, Inc New York, NY. pp. 582--594.

\hypertarget{ref-punt2016management}{}
Punt, A.E., Butterworth, D.S., Moor, C.L., De Oliveira, J.A., and
Haddon, M. 2016. Management strategy evaluation: Best practices. Fish
Fish. Wiley Online Library.

\hypertarget{ref-schnute1995influence}{}
Schnute, J.T., and Richards, L.J. 1995. The influence of error on
population estimates from catch-age models. Can. J. Fish. Aquat. Sci.
\textbf{52}(10): 2063--2077. NRC Research Press.

\hypertarget{ref-smith1994management}{}
Smith, A. 1994. Management strategy evaluation: The light on the hill.
Population dynamics for fisheries management: 249--253. Proc. Australian
Society for Fish Biology Workshop, Australian Society for Fish Biology,
Perth.

\hypertarget{ref-wetzel2017performance}{}
Wetzel, C.R., and Punt, A. 2017. The performance and trade-offs of
alternative harvest control rules to meet management goals for us west
coast flatfish stocks. Fisheries research \textbf{187}: 139--149.
Elsevier.



\end{document}